\section{Algorithme}
\subsection{Présentation de l'algorithme}
L'algorithme que nous souhaitons implémenter, l'algorithme d'exclusion mutuelle en distribué de Lamport, a pour but d'exécuter une section du code de manière exclusive entre différents hôtes connectés entre eux (section critique).\\

Le principe de l'algorithme se base sur la présence d'une horloge, celle que nous utilisons est une horloge logique scalaire. Les événements suivants déterminent de quelle manière l'horloge va être mise à jour :

\begin{description}
    \item[événement local] $HL_i++$
    \item[envoi message] $HL_i++$ et $EL_m = HL_i$
    \item[réception message] $HL_i = 1 + max(HL_i, EL_m)$
\end{description}

\noindent \textbf{N.B.} $HL_i$ horloge logique de l'hôte $i$, et $EL_m$ estampille du message $m$.\\

Lorsqu'un des hôtes souhaite entrer en section critique, celui-ci ajoute sa requête dans une file et l'envoi à tous les autres hôtes. Il attend ensuite la réponse de ces derniers. Il ne peut entrer en section critique que si tous les hôtes ont répondu, et si sa requête est la première dans la file. Une fois la section critique exécutée l'hôte actuel retire sa requête de la file, et un message de libération est envoyé à tous les autres afin de les prévenir d'également retirer la première requête de leur file.

\subsection{Initialisation}
Avant d'entrer dans la boucle principale, les différentes variables servant à l'algorithme doivent être initialisées.

\begin{lstlisting}
int logical_clock = 0; // horloge logique scalaire
int responses = 0; // compteur de réponses reçues après envoi d'une requête
int state = STATE_NOTHING; // état courant du programme
node* queue = NULL; // liste chaînée permettant de stocker les requêtes de manière triée
\end{lstlisting}

\subsection{Récupération d'un message \& traitement}
La première étape d'une itération dans la boucle principale est de récupérer un message d'un des hôtes et de le traiter.  Une connexion entrante est d'abord acceptée avec \textbf{accept()} (configuré en non bloquant précédemment). Le message est ensuite récupéré à l'aide de \textbf{receive\_message()}. L'horloge logique est ensuite mise à jour à l'aide de l'estampille du message reçu ($HL_i = 1 + max(HL_i, EL_m)$). La chaîne de caractères contenue dans le message reçu détermine ensuite quelle action doit être effectuée :

\begin{description}
    \item[request] Màj de l'horloge ($+1$), et envoi d'une réponse à l'hôte ayant envoyé la requête.
    \item[response] Incrémentation du compteur de réponses.
    \item[free] Retrait du premier élément de la file.
\end{description}

\begin{lstlisting}[caption={Récupération d'un message \& traitement}]
int s_client = accept(s_listen, NULL, NULL);
if (s_client > 0) {
    message *msg = receive_message(s_client);
    close(s_client);
    if (msg == NULL) {
        exit(EXIT_FAILURE);
    }

    logical_clock = 1 + max(logical_clock, msg->timestamp);

    printf("Host(%d) - Clock(%d) - Received (from %d) : %s\n",
        cur_host_id, logical_clock, msg->host_id, msg->str
    );

    if (strcmp(msg->str, "request") == 0) {
        message* msg_response;
        int recipient = msg->host_id;

        insert_message(&queue, msg);
        logical_clock++;
        print_messages_linked_list(queue);

        msg_response = create_message(
            cur_host_id, logical_clock, "response"
        );

        send_message(argv[2 + recipient], atoi(argv[1]) + recipient,
            msg_response);
        printf("host(%d) - Clock(%d) - Sent response (to %d)\n",
            cur_host_id, logical_clock, recipient);
        free_message(msg_response);
    }
    else if (strcmp(msg->str, "response") == 0) {
        responses++;
        free_message(msg);
    }
    else if (strcmp(msg->str, "free") == 0) {
        pop(&queue);
        print_messages_linked_list(queue);
        free_message(msg);
    }
}
\end{lstlisting}

\subsection{Mise à jour de l'état}
Une fois le message reçu est traité, nous souhaitons changer l'état dans lequel se trouve le programme. Les trois états possibles sont les suivantes :\\

\begin{description}
    \item[STATE\_NOTHING] Le programme n'a pas d'action en cours.
    \item[STATE\_WAITING] Une requête a été envoyée aux autres hôtes, le programme est en attente de réponse de ces derniers. Il attend également que sa requête soit la première dans la file.
    \item[STATE\_CRITICAL\_SECTION] Le programme est en cours d'exécution de sa section critique.
\end{description}
\

Trois opérations liées à la section critique peuvent être effectuées. celles-ci sont effectués à la suite (voir listing \ref{lst:rand1}) pour essayer de mettre à jour l'état du programme.\\

\begin{description}
    \item[try\_request\_cs()] envoie des requêtes pour entrer en section critique aux autres hôtes si l'hôte actuel n'en a pas déjà envoyé (\emph{STATE\_NOTHING}). La requête est également insérée dans la file.

    \item[try\_enter\_cs()] entre en section critique si les conditions requises sont remplies : une requête a été envoyée aux autres hôtes (\emph{STATE\_WAITING}), la première requête de la file appartient à l'hôte courant, et tous les autres hôtes ont envoyé une réponse.

    \item[try\_leave\_cs()] Sort de la section critique si le programme y est entré (\emph{STATE\_CRITICAL\_ SECTION}).
\end{description}

\begin{framehint}
\textbf{try\_enter\_cs()} et \textbf{try\_leave\_cs()} ont été placés à la suite, ce qui résultera toujours en \emph{Begin critical section} suivi de \emph{End critical section} tout de suite après. Seul l'appel \textbf{try\_request\_cs()} est actuellement aléatoire (voir listing \ref{lst:rand1}), cependant les trois appels peuvent être rendus aléatoires (voir listing \ref{lst:rand2}) afin de tester plus en détail le bon fonctionnement de l'implémentation.\\

\begin{lstlisting}[label={lst:rand1}, caption={Changement d'état aléatoire 1}, numbers=none]
if ((rand() % nhosts) == cur_host_id) {
    try_request_cs(&state, &logical_clock, nhosts, argv, &queue);
}
try_enter_cs(&state, &responses, &logical_clock, nhosts, queue,
    cur_host_id);
try_leave_cs(&state, &logical_clock, nhosts, argv, &queue);
\end{lstlisting}

\begin{lstlisting}[label={lst:rand2}, caption={Changement d'état aléatoire 2}, numbers=none]
if ((rand() % nhosts) == cur_host_id) {
    try_leave_cs(&state, &logical_clock, nhosts, argv, &queue);
    try_request_cs(&state, &logical_clock, nhosts, argv, &queue);
    try_enter_cs(&state, &responses, &logical_clock, nhosts, queue,
        cur_host_id);
}
\end{lstlisting}
\end{framehint}

\begin{lstlisting}[caption={Fonctions de mise à jour d'état}]
void try_enter_cs(int* state, int* responses, int* logical_clock, int nhosts,
    node* queue, int cur_host_id)
{
    if (*state == STATE_WAITING &&
        queue->msg->host_id == cur_host_id &&
        *responses == nhosts - 1)
    {
        *responses = 0;
        *state = STATE_CRITICAL_SECTION;
        (*logical_clock)++;

        printf("Host(%d) - Clock(%d) - Begin critical section\n",
            cur_host_id, *logical_clock);

        //sleep(2); // DEBUG ONLY
    }
}

void try_request_cs(int* state, int* logical_clock, int nhosts, char* argv[],
    node** queue)
{
    if (*state == STATE_NOTHING) {
        int cur_host_id = get_host_pos(nhosts, argv);
        message *request_msg;

        (*logical_clock)++;

        request_msg = create_message(cur_host_id, *logical_clock, "request");
        insert_message(queue, request_msg);

        send_message_all(nhosts, cur_host_id, argv, request_msg);

        printf("host(%d) - Clock(%d) - Sent request\n",
            cur_host_id, *logical_clock);
        print_messages_linked_list(*queue);

        *state = STATE_WAITING;
    }
}

void try_leave_cs(int* state, int* logical_clock, int nhosts, char* argv[],
    node** queue)
{
    if (*state == STATE_CRITICAL_SECTION) {
        int cur_host_id = get_host_pos(nhosts, argv);
        message *free_msg;

        (*logical_clock)++;

        free_msg = create_message(cur_host_id, *logical_clock, "free");
        send_message_all(nhosts, cur_host_id, argv, free_msg);
        free_message(free_msg);

        pop(queue);

        printf("Host(%d) - Clock(%d) - End critical section\n",
            cur_host_id, *logical_clock);
        print_messages_linked_list(*queue);

        *state = STATE_NOTHING;
    }
}
\end{lstlisting}
