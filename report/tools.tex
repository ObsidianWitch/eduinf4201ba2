\section{Outils nécessaires à l'algorithme}
\subsection{Messages}
Une structure de message a été définie afin de faire communiquer les différents hôtes du système distribué. Un message est caractérisé par l'id de l'hôte (\emph{host\_id}) duquel il provient, son estampille (\emph{timestamp}) ainsi que la chaîne de caractères du message (\emph{str}). Cette structure ainsi que les fonctions qui lui sont liées sont définies dans \emph{messages.c/.h}.\\

\begin{lstlisting}
typedef struct message {
	int host_id;
	int timestamp;
	char* str;
} message;
\end{lstlisting}
\

Les fonctions \textbf{create\_message()} et \textbf{free\_message()} permettent respectivement de créer et libérer un message. La création de message fait notamment une copie de la chaîne de caractères passée en paramètre afin de s'assurer que l'on puisse la libérer par la suite.

\begin{framewarning}
Si la copie n'avait pas été effectuée, des problèmes auraient pu se produire en passant en paramètre une chaîne provenant de la stack, puis en libérant le message à l'aide de \emph{free\_message()}.
\end{framewarning}

\begin{lstlisting}
message* create_message(int host_id, int timestamp, const char* str) {
	message* msg = malloc(sizeof(message));

	msg->host_id = host_id;
	msg->timestamp = timestamp;
	msg->str = malloc(strlen(str) + 1);
	strcpy(msg->str, str);

	return msg;
}
\end{lstlisting}
\

La fonction \textbf{send\_message()} se contente de transformer un message en une chaîne de caractères à l'aide de \textbf{pack\_message()}. Cette chaîne de caractères est constitué de tous les champs du message séparés par des tirets. Une fois cette étape effectuée, le message est envoyé avec \textbf{send\_complete\_host()} défini dans \emph{socket\_tools.c/.h}.\\

\begin{lstlisting}
char* pack_message(message* msg) {
	char* buf = malloc(BUFFER_LEN);

	int status = snprintf(buf, BUFFER_LEN,
		"%d - %d - %s",
		msg->host_id,
		msg->timestamp,
		msg->str
	);

	if (status == 0) {
		free(buf);
		return NULL;
	}

	return buf;
}
\end{lstlisting}
\

La réception d'un message effectue l'inverse, une chaîne de caractères est tout d'abord reçu à l'aide de \textbf{recv\_complete()}, puis celle-ci est parsée avec \textbf{unpack\_message()} afin de la transformer en message.\\

\begin{lstlisting}
message* unpack_message(char* msg) {
	message* unpacked_message = malloc(sizeof(message));
	int host_id, timestamp, status;
	char* buf = malloc(BUFFER_LEN);

	status = sscanf(msg, "%d - %d - %s", &host_id, &timestamp, buf);
	if (status != 3) {
		return NULL;
	}

	unpacked_message->host_id = host_id;
	unpacked_message->timestamp = timestamp;
	unpacked_message->str = buf;

	return unpacked_message;
}
\end{lstlisting}
\

\subsection{Liste chaînée}
Une liste chaînée a également été créée, celle-ci étant nécessaire à l'algorithme d'exclusion mutuelle en distribué de Lamport. Elle permet d'ordonner les requêtes de section critique par estampille et par id d'hôte en cas d'estampilles identiques (ordre total mais artificiel).\\

La structure de liste chaînée pour les messages ainsi que les fonctions liées à la manipulation de celle-ci sont définies dans \emph{message\_linked\_list.c/.h}.

\begin{lstlisting}[caption=Structure de noeud de la liste chaînée]
typedef struct node {
    message* msg;
    struct node* next;
} node;
\end{lstlisting}
\

Les fonctions \textbf{create\_node()} et \textbf{free\_node()} permettent respectivement de créer et libérer un noeud de la liste chaînée. Un noeud nouvellement créé possède un message et son noeud suivant est \emph{NULL}. La libération d'un noeud libère également le message contenu dans celui-ci. Tous les noeuds d'une liste chaînée alloués en mémoire peuvent être libérés à l'aide de \textbf{free\_linked\_list()}.\\

Un noeud peut être inséré à l'aide de \textbf{insert\_message()}, cette fonction se charge de créer un noeud et d'insérer celui-ci de manière ordonnée dans la liste. Pour effectuer cette insertion plus facilement, la fonction \textbf{nodecmp()} permettant de comparer deux noeuds est utilisée. Celle-ci retourne un entier inférieur, égal ou supérieur à 0 si n1 est respectivement inférieur, égal ou supérieur à n2. Cette comparaison se base d'abord sur l'estampille du message, puis sur l'id de l'host si les estampilles sont égales. A l'aide de cette fonction nous traitons donc les cas suivants:\\

\begin{itemize}
	\item La liste chaînée est vide\\ $\rightarrow$ le nouveau noeud devient le premier noeud de la liste chaînée.
	\item Le nouveau noeud est inférieur au noeud présent au début de la liste chaînée\\ $\rightarrow$ on insère le nouveau noeud au début de la liste à l'aide de \textbf{insert\_node\_front()}.
	\item Le nouveau noeud doit être placé après le premier noeud de la liste chaînée\\ $\rightarrow$ on parcourt la liste chaînée jusqu'à trouver le noeud qui doit précéder le nouveau noeud, et on l'insère à cet endroit.
\end{itemize}
\

\begin{lstlisting}[caption=Insertion d'un noeud/message]
int nodecmp(const node* n1, const node* n2) {
    message *m1 = n1->msg;
    message *m2 = n2->msg;

    if (m1->timestamp < m2->timestamp) {
        return -1;
    }
    else if (m1->timestamp > m2->timestamp) {
        return 1;
    }
    else {
        if (m1->host_id < m2->host_id) {
            return -1;
        }
        else if (m1->host_id > m2->host_id) {
            return 1;
        }
        else {
            return 0;
        }
    }
}

void insert_node_front(node** linked_list, node* new_node) {
    node* tmp = *linked_list;
    *linked_list = new_node;
    new_node->next = tmp;
}

void insert_message(node** linked_list, message* new_msg) {
    node* new_node = create_node(new_msg);

    if (*linked_list == NULL) {
        *linked_list = new_node;
    }
    else if (nodecmp(new_node, *linked_list) <= 0) {
        insert_node_front(linked_list, new_node);
    }
    else if (nodecmp(new_node, *linked_list) > 0) {
        node* cur_node = *linked_list;
        node* next_node = cur_node->next;

        while(next_node != NULL && nodecmp(new_node, next_node) > 0) {
            cur_node = cur_node->next;
            next_node = cur_node->next;
        }

        new_node->next = next_node;
        cur_node->next = new_node;
    }
}
\end{lstlisting}
\

Enfin, le premier noeud d'une liste chaînée peut être supprimé à l'aide de \textbf{pop()}. Cette fonction se contente de définir le noeud suivant comme le premier noeud, et de libérer l'ancien noeud.

\begin{lstlisting}[caption={pop(), supression du premier noeud}]
void pop(node** linked_list) {
    if (*linked_list != NULL) {
        node* tmp = (*linked_list)->next;
        free_node(*linked_list);
        *linked_list = tmp;
    }
}
\end{lstlisting}
