\section{Introduction}
L'objectif de ce TP est d'implémenter l'algorithme d'exclusion mutuelle de Lamport étudié en cours. Une base nous a été fournie à partir de laquelle nous allons effectuer l'implémentation. Des structures de message et de liste chaînée seront notamment nécessaire afin de mettre en place l'algorithme.

\subsection{Arborescence}
\noindent Ci-dessous l'arborescence de fichiers du TP avec pour certains d'entre-eux une brève description.

\begin{framed}
\dirtree{%
.1 TP\_Dist.
.2 src.
.3 host\_tool.c/.h - Fonctions fournies avec le squelette du programme, permettent entre autre d'envoyer/recevoir des messages de synchronisation et de récupérer l'id de l'hôte courant..
.3 message\_linked\_list.c/.h - Structure de liste chaînée pour les messages et fonctions permettant de manipuler celle-ci (comparaison, ajout, retrait de noeuds)..
.3 message.c/.h - Structure de message et fonctions permettant de manipuler celle-ci (création, envoi, réception)..
.3 socket\_tools.c/.h - Outils permettant de faciliter la manipulation de sockets (initialisation d'une socket serveur/cliente, envoi et réception de données) (provient en majorité du TP précédent)..
.3 TP-DIST-CL.c.
.3 TP-DIST.c.
.2 Vagrantfile.
.2 Vagrantprov.sh.
}
\end{framed}

\subsection{Instructions de compilation}
Ci-dessous les instructions pour compiler l'ensemble des exercices. L'utilitaire cmake \cite{cite:cmake} est nécessaire afin de créer le Makefile.

\begin{mdframed}[backgroundcolor=lightblue, linecolor=darkblue]
	mkdir build\\
	cd build\\
	cmake ..\\
	make
\end{mdframed}
